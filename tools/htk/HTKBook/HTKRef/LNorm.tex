%
% Gareth Moore  27/03/02
%

\newpage
\mysect{LNorm}{LNorm}

\mysubsect{Function}{LNorm-Function}

\index{hlmcopy@\htool{LNorm}|(}
The basic function of this tool is to renormalise language models,
optionally pruning the vocabulary at the same time or applying cutoffs
or weighted discounts.

\mysubsect{Use}{LNorm-Use}

\htool{LNorm} is invoked by the command line
\begin{verbatim}
   LNorm [options] inLMFile outLMFile
\end{verbatim}
This reads in the language model {\tt inLMFile} and writes a new
language model to {\tt outLMFile}, applying editing operations
controlled by the following options.  In many respects it is similar
to \htool{HLMCopy}, but unlike \htool{HLMCopy} it will always
renormalise the resulting model.

\begin{optlist}

  \ttitem{-c n c} Set the pruning threshold for $n$-grams to $c$. 
	Pruning can be applied to the bigram and higher
	components of a model ($n$>1). The pruning procedure will keep only 
	$n$-grams which have been observed more than $c$ times. Note
	that this option is only applicable to count-based language 
        models.

  \ttitem{-d f}  Set weighted discount pruning for \texttt{n}-gram
        to \texttt{c} for Seymore-Rosenfeld pruning. Note that this
        option is only applicable to count-based language models.
  
  \ttitem{-f s} Set the output language model format to {\tt s}.
        Possible options are {\tt TEXT} for the standard ARPA-MIT
	LM format, {\tt BIN} for Entropic {\em binary} format and 
        {\tt ULTRA} for Entropic {\em ultra} format.
        
  \ttitem{-n n} Save target model as $n$-gram.

  \ttitem{-w f} Read a word-list defining the output vocabulary from
	{\tt f}. This will be used to select the vocabulary for 
	the output language model.

\end{optlist}
\stdopts{LNorm}

\mysubsect{Tracing}{LNorm-Tracing}

\htool{LNorm} supports the following trace options where each
trace flag is given using an octal base
\begin{optlist}
   \ttitem{00001} basic progress reporting.
\end{optlist}
Trace flags are set using the \texttt{-T} option or the  \texttt{TRACE} 
configuration variable.
\index{hlmcopy@\htool{LNorm}|)}
